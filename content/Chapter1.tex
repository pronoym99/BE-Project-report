\chapter{Introduction}
\thispagestyle{empty} % no page number for introduction 

\setcounter{page}{2} % remember counting now starts from 2
\pagenumbering{arabic} % style changes back to arabic

\section{Project name and title}
The name of our project is CNC machine based complete PCB assembly line (milling).

\section{Problem statement}
Current factory-based PCB manufacturing targets large clients primarily for mass production. However, when it comes down to normal students, hobbyists, engineers mass production of PCBs is not what they expect. These small scale customers want fast but at the same time reliable and quality output for their products. \par

To solve the same this project attempts to create a CNC based PCB milling machine primarily for such small scale customers. Standard home-based PCB manufacturing techniques waste a considerable amount of time in the revelation of copper tracks on the PCB by sequential steps including (but not restricted to) UV exposure, etching, engraving, cleaning etc. till all tracks on a typical copper-clad are fully revealed. Although for PCBs of constant dimensions (which small scale clients usually prefer) each of the aforementioned steps takes on an average of 3 to 3.5 minutes time. \par

Hence, the total time required to develop a full PCB comes out to be 4 to 6 hours and may extend into multiple days if the concerned machines for the above processes are working under heavy load from other clients. This leads to a complete product delivery time in excess of 1-2 days (per PCB). 


\section{Proposed solution}
It turns out that the processing time can be greatly reduced if we manage to target and reduce the time taken in those steps which are consuming a constantly large amount of time for every PCB irrespective of the design on it.  \par

Here the word design although a layman term refers to many things at once. 
\begin{enumerate}
    \item The total number of tracks on the PCB
    \item The complexity in which the tracks are arranged amongst themselves
    \item The types of tracks (in terms of drill depth, length, thickness, clearance etc.)
\end{enumerate}

The time taken for the UV exposure process, etching process etc. are invariant of the design (word used in the same context as above) configuration of the PCB in terms of (a), (b) and (c). \par

A solution would be viable if we manage to directly engrave the tracks on a PCB as soon as the schematic file (.sch) or the board file (.brd) for a particular design is generated.In standard factory settings, if the above step is completed the PCB can proceed directly for drilling and component placement.However, the scope of our project is restricted to the successful completion of the former step in minimum time.



\section{Scope and report contents}
Chapter 2 starts with a critical analysis of the previous works that have been done using CNC machines including (but not restricted to) their applications in the PCB manufacturing industry. It goes on to discuss how every unit of the manufacturing process is done by optimal usage of CNC based processes. Added to that it also discusses few papers, journals as well as some commercial CNC instruction manuals. \par

Chapters 3 to 5 give a very detailed explanation of the current investigation. Chapter 3 deals with the manufacturing and construction of the CNC machine. Chapter 4 mostly deals with the main electronics and driving circuitry used in this project. While Chapter 5 details the reader on the software methodology and concludes the explanation of the current investigation. All the required steps are given sequentially in the order in which were carried out. \par

The final two chapters conclude with the findings of the current investigation, the logical results and analysis associated with it as well as give an insight into the future work plan of the project. 