\chapter*{Abstract}

A CNC based PCB Milling Machine incorporates the plan and usage of a CNC (Computer Numerical Control) machine to make a PCB (Printed Circuit Board) in a particular set up. Creation of a PCB within the present market is extremely costly when the manufacture of testing circuits is taken into account. Along these lines, this CNC based PCB machine would be an efficient instrument through which circuits are often scratched and penetrated (or engraved) at an inexpensive rate. The creation of this CNC machine is to diminish the cost and multifaceted nature of machine. This venture manages the plan of a programmed scaled down CNC machine for PCB drawing and processing.
A PCB mill is a gadget that engraves out details from a Gerber file on a copper clad board with the end goal that it makes a Printed Circuit Board (PCB). PCBs are utilized in the field of electrical design to interface electrical parts to each other.

In CNC machine the program is present or loaded in the memory of a computer system. The software engineer can easily compose the codes and alter the projects according to their requirements. These projects can be utilized for various parts and they don’t need to be rehashed once more. The CNC machine offers more prominent adaptability and computational ability. New frameworks and components can be joined into the CNC controller essentially by reconstructing the unit.

The CNC machine involves a PC or a computation device in which the program is present for cutting of the metal from the job according to the given specifications. All the slicing forms that are to be completed and all last measurements are encouraged into the PC through a program (essentially G-code). The PC realizes what precisely is to be done and controls all the cutting procedures. CNC machine works like a robot, which must be bolstered with the program that it is supposed to take after every one of your directions. A subset of the normal machine devices that can keep running on the CNC are: drilling machine or milling machine.

The principle motivation behind these machines is to expel a portion of the metal to give it appropriate shape, for example, round, rectangular and so on, the software guides the machine apparatus to perform different machining operations according to the program of guidelines given by the user, all the CNC machines are intended to meet close correctness.

The CNCs have tremendous advancement potential and many machining procedures can be communicated by methods for basic operations, for example interpretations, revolutions and by controlling the status of the CNC head. Numerous CNCs are controlled utilizing a unique programming dialect, named G-Code.
